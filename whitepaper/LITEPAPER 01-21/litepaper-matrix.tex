% ****** Start of file apssamp.tex ******
%
%   This file is part of the APS files in the REVTeX 4.2 distribution.
%   Version 4.2a of REVTeX, December 2014
%
%   Copyright (c) 2014 The American Physical Society.
%
%   See the REVTeX 4 README file for restrictions and more information.
%
% TeX'ing this file requires that you have AMS-LaTeX 2.0 installed
% as well as the rest of the prerequisites for REVTeX 4.2
%
% See the REVTeX 4 README file
% It also requires running BibTeX. The commands are as follows:
%
%  1)  latex apssamp.tex
%  2)  bibtex apssamp
%  3)  latex apssamp.tex
%  4)  latex apssamp.tex
%
\documentclass[%
 reprint,
%superscriptaddress,
%groupedaddress,
%unsortedaddress,
%runinaddress,
%frontmatterverbose, 
%preprint,
%preprintnumbers,
nofootinbib,
%nobibnotes,
%bibnotes,
 amsmath,amssymb,
 aps,
%pra,
%prb,
%rmp,
%prstab,
%prstper,
%floatfix,
]{revtex4-2}

\usepackage{graphicx}% Include figure files
\usepackage{dcolumn}% Align table columns on decimal point
\usepackage{bm}% bold math
%\usepackage{hyperref}% add hypertext capabilities
%\usepackage[mathlines]{lineno}% Enable numbering of text and display math
%\linenumbers\relax % Commence numbering lines


\begin{document}

\preprint{APS/123-QED}

\title{MATRIX: DECENTRALIZED RESERVE CURRENCY}% Force line breaks with \\
%\thanks{A footnote to the article title}%

\author{MatrixLabs}
 \altaffiliation[ ]{Zurich, Switzerland.}%Lines break automatically or can be forced with \\
%\author{Second Author}%
% \email{Second.Author@institution.edu}
%\affiliation{%
% Authors' institution and/or address\\
% This line break forced with \textbackslash\textbackslash
%}%

%\collaboration{MUSO Collaboration}%\noaffiliation
%
%\author{LightningLabs}
%\homepage{http://www.Second.institution.edu/~Charlie.Author}
%\affiliation{Luxemborg}%
%\affiliation{Third institution, the second for Charlie Author
}%
%\author{Delta Author}
%\affiliation{%
% Authors' institution and/or address\\
% This line break forced with \textbackslash\textbackslash
%}%

\date{\today}% It is always \today, today,
             %  but any date may be explicitly specified

\begin{abstract}

Matrix presents decentralized, over-collateralized and capital-efficient reserve protocol using Cosmos SDK. Matrix enables liquid, capital efficient convertibility between stable assets and collateral using front-running resistant oracle to achieve the swapping. 
\end{abstract}
\keywords{DAO, Reserve Currency}
\maketitle


\section{Introduction}\label{Section1}
Decentralized Finance stablecoins can help 1.7 billion adults who lack access to the banking system. Current stablecoin protocols remain incomplete. USDT and USDC, which have the largest market capitalization, are supported by centralized projects. 
Decentralized protocols that mint and burn stable assets with collateralized debt positions can be unstable with discretionary choices of parameters.
Stablecoin designs that are algorithmic, under-collateralized, and overly capital efficient have experienced bank-run situations and liquidity crises.

Currently, there is no protocol for a Cosmos-native decentralized reserve currency. Matrix protocol allows swaps at oracle value between speculative assets and collateral. The volatility of the collateral in the reserves is hedged using perpetual futures. 

Matrix achieves stability by transferring collateral volatility to third party users seeking leveraged long exposure with no funding rates. Matrix aims for over-collateralization by incentivizing two types of Liquidity Providers (\textbf{LPs}): Leverage Agents (LAs) and Insurance Agents (IAs). These agents bring extra collateral to the protocol and help protect against liquidation risks.

\subsubsection*{Summary - Paper Sections} 
\begin{itemize}
  \item \textbf{\ref{Section2} Neo Token}: Explains the importance of the Neo token.
  \item \textbf{\ref{Section3} Liquidity Providers}: Introduces the two types of LPs in the Matrix Protocol.
  \item \textbf{\ref{Section4} Matrix DAO}: Presents the Matrix DAO, which is designed to incentivize LPs and drive both the value and utility of the Neo token.
  \item \textbf{\ref{Section5} Major Contributions}: Highlights the major contributions of the Matrix protocol.
\end{itemize}


%Appendix~\ref{Section6} outlines the core-modules in the design and implementation of the Matrix protocol.

\section{NEO Token}\label{Section2}

The goal of the Matrix protocol is to mint stable assets tradable on the Cosmos ecosystem. Matrix will mint NEO, a stablecoin pegged to USD that will use OSMO as collateral at genesis. Matrix will let users mint stablecoins with the collateral of their choice and exchange stablecoins for collateral at oracle value with minimal fees. From a user’s perspective, minting and burning NEO with the protocol will feel just like a swap. Depositing collateral to the protocol mints NEO, and withdrawing collateral from the protocol burns NEO.

\subsection{Mint and Burn}

\begin{enumerate}
\item 
  To mint NEO, a stable asset, a user must send Matrix whitelisted collateral. Oracles determine how many NEO tokens are minted and given to the user. For example, if the oracle price for OSMO is \$10 and the protocol has a 0.3\% transaction fee, a user depositing 1 OSMO will receive 9.7 NEO from the Matrix.
\item 
  A user that intends to swap NEO for collateral specifies the desired kind of whitelisted collateral and exchanges the assets based on oracle price. The NEO received by the protocol is burnt. 
\item 
  The mint and burn transactions are executed without slippage on the price, regardless of the size of the transaction. The reason there are fees on Matrix transactions is to remunerate LPs as they guarantee the robustness of the system.
\item 
  Direct arbitrage opportunities arise whenever NEO trades at a price that differs from its peg. If 1 NEO trades at a price above \$1. The incentive is to mint NEO for \$1 worth of collateral and then sell these newly minted NEO for more than \$1 on the market, reducing the price of NEO closer to \$1. Opportunities for arbitrage are reduced when there are transaction fees. 
\item 
  A stable seeker will use Matrix to convert collateral to NEO, however Matrix does not need over-collateralized loans.
\end{enumerate}

\subsection{Need for \textbf{LPs}}

  If the protocol has 1 OSMO worth \$10 and 10 NEO is minted, the collateral ratio is 100\%. If the price of OSMO decreases to \$5, the collateral ratio is now 50\%. 
  A user could now swap 5 NEO for 1 OSMO from Matrix. This would reduce the exchange collateral and leave the user with 5 NEO, lowering the intrinsic value of NEO below \$1.
  To enable swapping collateral against stablecoins and stablecoins against collateral, Matrix protocol aims for over-collateralization by incentivizing Liquidity Providers (\textbf{LPs}).


\section{Liquidity Providers (\textbf{LPs})}\label{Section3}

\subsection{Leverage Agents (\textbf{LAs})}

When a stable seeker gives 1 OSMO to the Matrix protocol, the protocol is subject to the volatility of OSMO. To insure the protocol against the volatility of the collateral backing the stablecoins, Matrix creates a way to transfer the volatility to other actors willing to get leverage on collateral: the Leverage Agents (\textbf{LAs}). 

\textbf{LAs} get perpetual futures from the Matrix protocol. These futures give LAs leveraged exposure on the collateral backing NEO. \textbf{LAs} insure Matrix against drops in collateral price by helping guarantee that the protocol has enough reserves to reimburse stable holders. The \textbf{LAs} bring some collateral, and choose the leverage ratio to insure the amount of collateral from stable seekers. 

Let's say that stable seekers bring $c_{\text{matrix}}$ collateral (e.g. OSMO) to the Matrix Protocol and LAs choose to cover an amount, $c_{LA}$. The quantity,
$$ \frac{ c_{\text{matrix}} + c_{LA} }{ c_{\text{matrix}} } ,$$

is the leverage multiplier for this LAs position because LAs receive gains or pay for losses from the covered colalteral, $c_{LA}$. The value of an LA position with the protocol is given by: 
$$ 
\text{position\_value} = c_{\text{matrix}}\cdot \left( 1 - \frac{\text{price\_initial}}{\text{price\_current}}  \right) + c_{LA}.
$$

This implies that an LA gets liquidated if the price drops to 
$$
\text{price\_current} =  \text{price\_initial} \times \left(\frac{ c_{\text{matrix}} }{ c_{\text{matrix}} + c_{LA} }\right).
$$

\paragraph*{Example:} 
If 1 OSMO is worth \$10, the \textbf{LAs} brings $c_{LA}$ = 10 and covers $c_{\text{matrix}}$ = 50, then the \textbf{LAs} takes the variation of 50 OSMO ($c_{\text{matrix}}$) with their 10 OSMO ($c_{LA}$). If OSMO goes to 12\$, the \textbf{LAs} will earn a leveraged return of $10 + 50 \times (1 - 10/12)$ = 18.33 OSMO. On the other hand, if the price of OSMO drops to $\frac{50}{60} \times 10 = \$8.33$, the \textbf{LA} will have lost her investment, which will be absorbed by the protocol to help it stay collateralized. \textbf{LAs} will not be able to back-any amount greater than the collateral brought by stable-seekers.

%  \footnote{Derivation on the constraints for \textbf{LAs}}. }. 

\paragraph*{\textbf{LAs} pay transaction fees when they enter and exit positions.} Matrix does not incentivize the \textbf{LAs} with funding rates like centralized exchanges. Entry and exit costs to the protocol depend on the coverage ratio on the collateral from stable seekers. Low coverage ratios make it expensive to enter/exit the protocol.  

\paragraph*{\textbf{LAs} positions cannot be modified.} A single address can own multiple \textbf{LA} positions within the same pool or within different pools of the protocol. \textbf{LA} positions are locked after each update to prevent \textbf{LAs} from taking advantage of potential oracle failures or advantages.

There may be cases where \textbf{LAs} fail to fully insure the protocol collateral brought by stable seekers. Matrix proposes a new type of liquidity provider to account for these temporary imbalances and serve as a buffer between stable-seekers and the leverage agents. These agents are called Insurance Agents (\textbf{IAs}).


\subsection{Insurance Agents (\textbf{IAs})}

Insurance Agents (\textbf{IAs}) act as a buffer for the moments when Leverage Agents do not fully insure the protocol’s reserves (collateral brought by users). \textbf{IAs} entrust Matrix with their liquidity so that they can accrue interest on it. The risk for an insurance agent is to incur slippage when the protocol is not well collateralized.

\subsubsection{Incentive Design}

Since \textbf{IAs} are taking a risk when lending money to over-collateralize the protocol, they should be incentivized for this risk:

\begin{enumerate}
\item 
  A fraction of the transaction fees obtained when minting or burning NEO are redistributed to IAs in proportion to how much they contribute to the protocol. \textbf{IAs}  will be able to stake their positions to receive governance tokens.
\item 
  \textbf{IAs} earn interests not only on the collateral they lent but also on the collateral brought by stable seekers\footnote{For the \textbf{IAs} to get the extra yield, part of the Matrix reserves need to be transferred for getting yield on it (i.e., lending to protocols). Suppose the protocol owns 1.5 OSMO out of which 1 comes from users who minted and 0.5 comes from \textbf{LPs}. If the whole protocol reserves are put into lending strategies, \textbf{LPs} will be receiving interests on 1.5 OSMO although they just brought 0.5 OSMO: they receive 3 times more interests than they would get by lending directly to other protocols. Because of the multiplier effect the Matrix protocol can guarantee higher yield to \textbf{IAs} than other lending platforms. The transaction fees, governance tokens and interests which the \textbf{IAs} receive is based on the over-collateralization ratio of the Matrix protocol.}.
\item 
  Not all the transaction fees and lending returns are given to IAs. The Matrix DAO decides on the distribution of the fees between the \textbf{LPs} and the treasury. 
\item 
  \textbf{IAs} receive tokens for contributing to a collateral pool. An \textbf{IA} that brings OSMO to the protocol receives sNEO for its share of the OSMO-NEO pool. The amount of sNEO received is based on the current exchange rate, which is dictated by an oracle price. On issuance of sNEO, an \textbf{IA} earns interest and rewards from (1) transaction fees that arrive to the pool and (2) interest from collateral being lent. \textbf{IAs} only receive transaction fees for the pools they contribute to. 
\item 
  \textbf{IAs} ensure the collateralization of the protocol in the situations when there are no perfect matches between users and \textbf{LAs}. If the protocol is under-collateralized, \textbf{IAs} incur slippage and may be at risk of not being able to get all of their money back. 
  For example, suppose that 100 NEO tokens are minted with 10 OSMO backing them as collateral. Thus,  1 OSMO is worth 10\$ and the protocol is fully collateralized. Then an \textbf{IA} brings 1 OSMO to the protocol. She will accrue fees from transactions and yield on the 11 OSMO available. In the event OSMO price drops to  \$8, the protocol would be under-collateralized and the \textbf{IA} would not be able to get all her money back when exiting the \textbf{IA} position. 
\item 
  With a slippage of 90\%, an \textbf{IA} willing to cash out 1 OSMO would only be able to get 0.1 OSMO. The slippage factor is a piecewise linear function of the collateral ratio, making the risks predictable for \textbf{IAs} while still incentivizing them to stay in the protocol. Having a continuous function is also necessary to limit front-running effects. The smaller the collateral ratio, the bigger the slippage. Above a certain collateral ratio (for instance 120\%), no slippage will be set for \textbf{IAs}.
\end{enumerate}

\section{Matrix DAO} \label{Section4}

Matrix DAO uses an incentive pendulum for the optimal allocation of the rewards (fees, yield & governance tokens) between the \textbf{LPs} and an Insurance Fund (\textbf{IF}) to make sure the protocol is stable and the users can get their collateral back during market stress events.

\subsection{Incentive Pendulum}

The objective of the design is to prevent the Matrix DAO from becoming \textbf{inefficient}, (majority of rewards to \textbf{LPs} and not enough funds in the (\textbf{IF}) for payout during bear markets) or \textbf{under-utilized} (majority of assets in the (\textbf{IF}), increasing bankruptcy risk due to under-collateralization). The on-chain redistribution of the assets and the rewards between the (\textbf{LPs}) and the (\textbf{IF}) is automated. The IF is a reserve fund created and seeded at genesis in the DAO to guarantee payouts to (\textbf{LPs}). At Genesis, the IF is seeded and the assets in the DAO are divided between the LPs and the IF. The genesis allocation to LPs depends on user validation and delegation. 

The DAO profit split is calculated as ($\Xi$) where $\Xi = \frac{LP + IF}{LP- IF}$ ; $LP$ = LP assets, $IF$ = IF assets. The DAO uses the inverse of the $\Xi$ to distribute the funds between LP and IF. The incentive pendulum monitors the distribution of the assets and rewards between the LP and the IF to make Matrix operate and allocate between the following states. 

Table(\ref{LP&IF}) provides the split between SF and IF at genesis.

\begin{table}
\begin{center}
\def~{\hphantom{0}}
\begin{tabular}{lccccc}
Scenario & LP Assets & IF Assets & $\Xi$ & LP Rewards & IF Rewards \\
\hline
Optimal & 67\% & 33\% & 3 & 67\% & 33\% \\
Under-utilized & 50\% & 50\% & - & 0 & 100\% \\
Inefficient & 100\% & 0 & 1 & 100\% & 0\\
\end{tabular}
\caption{Incentive Pendulum: Allocation of the capital between the LP and the IF, where $\Xi = \frac{LP + IF}{LP- IF}$ ; $LP$ = LP assets, $IF$ = IF assets. }
\label{LP&IF}
\end{center}
\end{table}

\textbf{Optimal}, $\Xi^{-1} = \frac{1}{3}$: The desired state, where 67\% of the assets are provided by LPs and 33\% are provided by the IF. The rewards from the protocol are split with the same proportions. 

\textbf{Under-utilized}, $\Xi^{-1}=0$ : An unsafe state. Here, both LPs and the IF provide 50\% of the assets each. 100\% of the rewards go to the IF (i.e. no rewards go to LPs). If the protocol is under-utilized, stakers have no incentive to validate the block-chain. This scenario needs to be explained in further detail to completely understand the design of the DAO; the balances of Matrix in the Insurance Fund is a function of the bankruptcy risk\footnote{At Genesis, IF will be initialized with a percentage of the Matrix tokens. Matrix tokens will be utilized to update the weighted capital allocation policy between the LP and IF periodically through on-chain governance (proposal and voting-based governance mechanism).}.

\textbf{Inefficient}, $\Xi^{-1}=1$: LPs provide 100\% of the assets and receive 100\% of the protocol rewards. When the protocol becomes inefficient, capital and rewards are re-allocated by the DAO between LPs and the IF to bring about additional investment yields, moving the protocol closer to optimality. The Matrix incentive pendulum will help maintain the protocol at near-optimal states. Ultimately, Matrix stakers can vote to re-capitalize the Insurance Fund.

\subsection{Governance}

A core principle of Matrix Protocol is to maintain its decentralized nature. The Matrix DAO will work on this objective from the beginning. Matrix DAO is responsible for helping the protocol realize its full potential, improving its design, and making a valuable building block in the DeFi space. The DAO will be responsible for parameter tuning, for deploying new stablecoins, for protocol upgrades and integrations, and voting to accept new collateral for a given stablecoin.

\begin{enumerate}
\item The governance token of the protocol is vNEO. The idea with Matrix’s governance token is to make the protocol decentralized and to distribute vNEO to community members that most use the protocol and/or collateralize it as IAs or LAs. The exact token distribution is likely to change. 
\item Governance tokens will be distributed through staking contracts and through a bonding curve, letting people buy governance tokens using the protocol’s stablecoins 
\footnote{The price at which governance tokens are sold is an increasing function of the number of tokens already sold through this mechanism. The bonding curve is a cheap way to increase the collateral ratio of the protocol. Governance token distribution will be used to incentivize people to contribute more to one people than another. It may not be directly activated at protocol launch.}. 

\item 
  The only situation in which the total supply is likely to inflate is when governance needs to be able to sell more tokens through the bonding curve to collateralize the protocol.
\item 
  Governance can choose to use the bonding curve to make it cheapr to burn stablecoins against vNEO, enabling the protocol to collateralize itself. There is a collateral settlement process that freezes transactions at the current oracle value, allowing stable holders, IAs, and LAs to claim the collateral that’s owed to them 
  \footnote{Details of the design of the incentive pendulum and the distribution/bonding of the governance token will be published.}.
\end{enumerate}


\section{Major Contributions}\label{Section5}

Matrix is a decentralized protocol designed to create reserve currency in a capital efficient way. It aims to improve over centralized designs as well as over over-collateralized and under-collateralized decentralized designs. Matrix innovates by proposing full convertibility at a 1:1 rate between stable assets and collateral. Collateral can always be swapped against stablecoins, and stablecoins can always be swapped against collateral at oracle value. The protocol involves 3 agents which all benefit from Matrix: Stable Seekers and Holders who issue and use stable assets, (\textbf{LAs}), who get perpetual futures from the protocol while insuring it against the variability of the collaterals’ prices and (\textbf{IAs}), who help the protocol have enough liquidity at all times, even when the protocol is not fully covered by LAs.





%\section{Introduction}
%
%The global sports betting market as reported in~\cite{Bire82},~\cite{epr} is projected to grow by \$ 130 billion during 2020-2024, projected at a 10\% CAGR during the forecast period. It is a market that suffers from geographic isolation due to regulatory constraints, centralized operations that are inherently flawed, and lack of access and payment transparency. Yet, this market as an asset-class is unparalleled on a risk adjusted return on capital and diversification.
%
%From a bettor’s perspective, centralization within existing sports betting systems gives an unnecessary informational advantage to the system operators. These platforms require bettors to organize through a trusted third-party for settlement of the bets, with wait times of up to 2 weeks for settlement of those payments. These entities are compromised by their own business interests and goals, to the detriment of the bettor.
%
%In particular, the need to trust a third-party authority (\textbf{house}) and the accumulation of power that follows, leads to: \textit(a) Vulnerability to odds manipulation, non-payment, or delayed payment; \textit(b) Exploitation of identity and betting pattern data collected from bettors; \textit(c) Total and irrecoverable loss of funds due to assets being frozen.
%
%For the first time in human history, users of sports betting platforms can act in whatever capacity they think will generate the best return.  Users can themselves become the house, one that takes wagers and pays odds real time, based on a transparent and fair oracle, and is ultimately governed by the users themselves.  Alternatively, bettors can act as traditional bettors, making wagers with Saage’s Blockchain based betting, which provides trustless and instant finality in bets settlement, removing need for a trusted third-party. 
%
%The rest of the paper is organized as follows. Section~\ref{Section2} explains the importance of the Saage token. Section~\ref{Section3} explains the design and implementation of the Saage DAO. Section~\ref{Section4} explains the modular Saage architecture: (1) the Blockchain layer; (2) Communication layer; and (3) Applications layer; and how the interaction between the various layers are designed to drive value and utility of the Saage token. 
%
%\section{Saage Token}\label{Section2}
%Saage is the native token that secures and operates the Saage ecosystem. The Saage token release schedule is determined at genesis and published publicly\footnote{The Saage inflation schedule is available at \url{https://saage.io/password.html}.}. 
%Saage token generates utility and value in the following categories. Saage token is the settlement and medium of exchange for the Saage ecosystem. Saage token is the reward asset for Stakers and Bettors. Saage token is used for validating the transactions associated with on-chain bet placement and voting needed by DVM. Saage token is used to guarantee payouts by the Insurance Fund (IF). Saage token is needed to participate in on-chain governance (voting on the staking parameters, voting using DVM, change in governance, deciding on fees and rewards in the system). 
%
%\section{Saage DAO} \label{Section3}
%
%Saage DAO acts as a bookmaker for any Saage-listed sporting event guaranteeing payout to the betters. Staking pool (SP) in the DAO provides liquidity to the Saage betting system, while the Insurance Fund (IF) acts as the DAO’s payout fund.   Saage DAO’s design enables 24x7 sports betting, guaranteeing full transparency in the settlement and the payment of bets of any size. Reserves in the DAO use an incentive pendulum for the optimal allocation of the assets as well as rewards between the SP and IF. 
%
%\subsection{Incentive Pendulum}
%The objective of the design is to prevent the protocol from becoming \textbf{inefficient}, (majority of assets in SP and not enough funds in the IF for payout to encourage bettors) or \textbf{unsafe} (majority of assets in the IF, increasing bankruptcy risk). The on-chain redistribution of the assets and the rewards between the SP and the IF is automated. The DAO receives inflation rewards and fees generated from betting activity, on-chain governance, odds listing, and on-chain data verification services. When a user bets on any Saage-listed sports event, the bet amount is locked in the IF. The 
%IF is a reserve fund created and seeded at genesis in the DAO to guarantee winning payouts and profit from the bettors’ losses.
%
%At Genesis, the assets in the DAO are divided between the SP and the IF, where the IF is seeded at genesis, and the SP are contingent on user validation and delegation. The IF profit split generated through wagering is calculated as ($\Xi$) where $\Xi = \frac{SP + IF}{SP- IF}$ ; $SP$ = SP assets, $IF$ = IF assets. The DAO uses the inverse of the $\Xi$ to distribute the funds between SP and IF. The incentive pendulum monitors the distribution of the assets and rewards between the SP and the IF to make Saage operate and allocate between the following states. Table(\ref{SP&IF}) provides the split between SF and IF at genesis.
%%
%%\begin{table}
%%\begin{center}
%%\def~{\hphantom{0}}
%%\begin{tabular}{lccccc}
%%Scenario & SP Assets & IF Assets & $\Xi$ & SP Rewards & IF Rewards \\
%%\hline
%%Optimal & 65\% & 35\% & 3 & 65\% & 35\% \\
%%Unsafe & 50\% & 50\% & - & 0 & 100\% \\
%%Inefficient & 100\% & 0 & 1 & 100\% & 0\\
%%\end{tabular}
%%\caption{Incentive Pendulum: Allocation of the capital between the SP and the IF. }
%%\label{SP&IF}
%%\end{center}
%%\end{table}
%
%\textit(a) \textbf{Optimal} This is the desired state, where SP has 65\% of the assets and the IF has 35\% of the assets. In such a situation, the system income is distributed as follows: 33\% for the IF and 67\% for the SP. 
%\textit(b) \textbf{Unsafe} The system may become unsafe where SP has 50\% of the assets and the IF has 50\% of the assets. In such a situation, the system income is distributed as follows: 100\% for the IF and 0\% for the SP. This is a scenario where the stakers have no incentive to validate the block-chain and the inflation rewards~(\ref{appB}) can not be properly tied to restore the balance, returning the system to the optimal state.\footnote{A detailed numerical investigation of the incentive pendulum, staking economics, and risk of the Insurance fund used in Saage will be presented and validated by a third party auditor. The report would provide guidelines for the distribution of the profit from the IF to the Saage DAO based on the total bets placed in Saage over 90 days}. 
%
%
%The unsafe scenario needs to be explained in further detail to completely understand the design of the DAO; the balances of Saage in the Insurance Fund is a function of the betting activity and bankruptcy risk, explained in detail in~\ref{slushfund}. At Genesis, IF will be initialized with a percentage of the Saage tokens. Saage tokens will be utilized to update the weighted capital allocation policy between the SP and IF periodically through on-chain governance (proposal and voting-based governance mechanism). 
%
%\textit(c) \textbf{Inefficient} The system can also become inefficient, where SP has 100\% of the assets and the IF has 0\% of the assets. In such a situation, the system income is distributed as follows: 0\% for the IF and 100\% for the SP. This situation would result in payouts for adverse bet settlements coming from balances in the SP. The expectation is that the Saage DAO would be performing such that capital and rewards will be re-allocated between SP and IF to pursue maximum yield, balancing the imbalance. The Saage incentive pendulum will help maintain the protocol at near-optimal states.  Ultimately Saage stakers can vote to re-capitalize the Insurance Fund.
%
%%\subsection{Design of IF}\label{slushfund}
%%
%%The Insurance Fund monitors the odds of being bankrupt and provides liquidity to guarantee full payment to winning bettors in all scenarios. The IF is initialized with a certain amount of Saage tokens\footnote{Inflation dynamics of the Saage token is presented in Appendix~\ref{appB}} equivalent to dollar amount: 
%%Saage is in optimal state: if the IF start with bankroll $\mathcal{W}_0$, and accepts bets which are a fixed fraction $f$ of $\mathcal{W}_0$ each time; with odds $d$ > 0\footnote{IF would receive multiple of d, when the IF wins or lose the multiple of 1}, with winning probability $p$ at each bet and losing probability  $q = 1 - p$.
%%
%%Besides monitoring the odds for betting on-chain, IF employs two threshold levels for balancing the reserve: \textbf{baseline cap (lower threshold)} and \textbf{max line cap (upper threshold)}. Baseline threshold occurs when IF is on a losing streak, with the bettors taking advantage of the excessive imbalance in the odds of the Saage- listed sporting events. During such events IF reserve is significantly reduced. The IF enters a protective mode, balances the decision-making ability of the on-chain analytics to stop accepting bets. The Insurance Fund will wait for the odds imbalance to improve and actively tries to off-load the one-sided risk it has accumulated. The system automatically adjusts the Profit and Loss distribution policy if the baseline threshold is met. This ensures that IF  will guarantee winner payouts in the case of unfavorable events. 
%%
%%When IF reserve exceeds the max line cap, Saage on-chain governance\footnote{Work in progress: Mathematical analysis of the risk offloading of the DAO and the computation max line cap value taking into consideration the total risk for a given time frame.} will be used to implement policies to permanently burn the excess amount from the reserve. Initially the parameters deciding the token buyback schedule is given in Appendix~(\ref{appA}). 
%
%\subsection{Blockchain Implementation} 
%Saage DAO performs as a SP and as an IF. The SP is implemented as a separate module using Cosmos-sdk named DAO settlement and uses the account module for the settlement of the bets. The interaction of the SP modules with other modules on Cosmos-sdk is detailed below: 
%
%During bet placement, the blockchain bet storage module keeper interface \textit{AppendBet} will invoke the DAO settlement and the keeper interface \textit{TransferBetStake}. This invokes the call to the Cosmos SDK in build bank module \textit{SendCoinsFromAccountToModule} to transfer the fund into the Dao account. To guarantee the payout to the winning bettors account \textit{Payout} module invokes the Bank module \textit{SendCoinsFromModuleToAccount}. Bet storage module processes the outcome in determining the winners and invokes the payout to the winner by calling Dao settlement module Payout function, which invokes the Cosmos SDK Bank module \textit{SendCoinsFromModuleToAccount} interface. Bet storage module updates the state of the bet using encrypted data.  In Phase 1, the payout would happen without using the DVM voting, subsequently in Phase 2 the payout is invoked using DVM\footnote{The voting mechanism is detailed in \textbf{\textit{How DVM works}} section, as it comes at the end of the DVM reveal phase.} Saage oracle\footnote{Separate whitepaper: Saage is powered by an in-house oracle pre-processing odds for Saage- listed sports events feeding the data to Saage. During initial phase, Saage will subscribe to betting odds from reputed sports odds providers and bookmakers and bring it on-chain~\cite{Oracle}.} provides the odds to determine the outcome of the betting event. 
%
%\section{Blockchain Layer}\label{Section4}
%
%\subsection{Saage Betting Cycle} 
%Betting on Saage is a sequential process from the creation of betting events to the payout. We outline the full decentralization and features deployment on Saage over three phases.
%
%\textbf{Betting Event Creation} In Phase1, Saage will provide odds for sports events from all the reputed third-party betting providers using a customized Oracle solution. In Phase 2, Saage token holders will be allowed to become additional providers of odds for Saage-listed sports events. In Phase 3; only odds provided by Saage token holders would be allowed for Saage listed sports events to celebrate full de-centralization and fairness in the settlement of the bets. \textbf{Bet Placement} All bets are recorded on-chain in Saage, where users can place bets on listed sports events. \textbf{Event Completion} Saage will fetch the event completion information from third-party providers. In Phase 2, Saage token holders can validate event completion data, progressively leading to decentralization. \textbf{Data Verification Mechanism\footnote{Data Verification Mechanism would be referred to as DVM}} In Phase 1, winners on the Saage platform would be paid instantly, since all betting outcomes would be based on pre-match odds provided at the start of the event. In Phase 2 and 3, Saage users will get to vote to on the settlement of the outcome of the events based on voting using DVM. The voting would be based on the amount of Saage tokens and be recorded on the Saage blockchain and provide the finality regardless who the counterparties are. \textbf{Bet Settlement} In Phase 1, the bet settlement would be instant due to the binary nature of the sports events listed on Saage and winners will receive their payout amount. In Phases 2-3 for complicated outcomes, the settlement of the outcome will be decided based on voting using the DVM. Saage will do the bet settlement for the betters who participated in the betting event. \textbf{Betting State Management} Saage has implemented the \textbf{bet storage module} to maintain bet states and to handle bets placed by users using the cosmos sdk module. 
%
%\subsection{How bet placement works?}
%Saage users can use a mobile app or the desktop version to place a bet on listed sports events. 
%
%The application will fetch the user's private key from the Saage in-app secure wallet, sign the user's bet selection with the private key, then broadcast the transaction to the Saage node bet storage transaction endpoint. As the protocol matures the integration of the ledger/trust wallet would be allowed. When the signed transaction arrives at the bet storage transaction endpoint, it will relay the message to the bet storage handlers for managing the creation of a bet object in this module. The state of the bet objects is maintained in the \textbf{bet storage module} in the keeper persistent key valet store. Bet storage keeper interface \textit{AppendBet} will process this bet placement transaction message of bet placement and the key valet store will be fetched and updated by the other custom module to maintain the bet objects’ state machine. Saage node'’s tendermint end point will receive the message, and then it will enter the consensus mechanism among other operating nodes. Once the consensus is done and the transaction is finalized, it will be delivered to the receiving node application blockchain interface (tendermint ABCI). ABCI will deliver the message to its destination bet storage module handler. This handler will verify the type of the message, and if it comes to know that it is bet message, it will call the bet storage keeper interface function \textit{AppendBet} to create a new bet object with the relevant input parameters like the bet amount, \textit{oddID}, the user’s public address, and a unique bet id. It will then store the bet on-chain in an encrypted format. The state of each individual bet is maintained on-chain. To lock the betting amount of the placed bet, \textit{betstorage} invoke the bet amount locking function \textit{TransferBetStake}, which transfers the Saage tokens from bettors to the DAO. 
%
%\subsection{Communication Layer}
%%
%The communication layer interfaces between the front and back end of the Saage. This layer converts and formats the odds data from Saage-listed sports events to the app and brings it on-chain. In Phase 1, Saage will list sports events with binary outcomes, so the settlement of bets would be instant. In subsequent phases, Saage will use the DVM, a community-driven, manual voting mechanism to decide on the settlement of the outcomes for the listed sports events. 
%
%\subsection{Application Layer} 
%
%Saage front end is designed to be available on iOS, Android, and desktops. The front end comes with a secure in-build wallet that supports most of the coins in the Cosmos ecosystem. The Saage application will display the upcoming sports events and the odds associated with them. Saage will display a sufficient range of betting odds from reputed odds providers, stakers, and internal, analytic community-driven models for managing Saage DAO risk. 

\bibliography{apssamp}

\end{document}

