% \documentclass{ar-1col}
% \usepackage[comma]{natbib}
\documentclass[11pt]{article}
\usepackage[review]{naacl2021}

\usepackage{url}
\usepackage{hyperref}
\hypersetup{
urlcolor=cyan, 
linkcolor=cyan,
filecolor=cyan,
}

% Standard package includes
\usepackage{amsmath,amssymb,amsfonts}
\usepackage{algorithmic}
\usepackage{graphicx}
\usepackage{textcomp}
\usepackage{xcolor}

\usepackage{multicol, float}
\usepackage{fancyhdr}
\usepackage{geometry}
\usepackage{graphicx, subfigure}
\usepackage[normalem]{ulem}
\useunder{\uline}{\ul}{}
\geometry{margin=1in}
\usepackage{subcaption}

\usepackage{multicol, float}
\usepackage{fancyhdr}
\usepackage{geometry}
\usepackage{graphicx, subfigure}
\usepackage[normalem]{ulem}
\useunder{\uline}{\ul}{}
\geometry{margin=1in}
\usepackage{subcaption}
\urlstyle{same}

\markboth{NIBIRU Labs}{NIBIRU}

% ---------------------------- Title

\title{
\begin{center}
 \includegraphics[scale=0.18]{nibi-logo-onwhite.png} 
\end{center} 
}

% ---------------------------- Document starts
\begin{document}

\maketitle
% \thispagestyle{fancy}
\lhead{Nibiru Labs, Zurich Switzerland}
\rhead{May, 2022}

% ----------------------------
% Abstract
% ----------------------------
\onecolumn
\begin{abstract}
NIBIRU is the first decentralized, permissionless derivatives platform leveraging its own AMM on Cosmos. NIBI is the staking and  governance token of the protocol.
\end{abstract}

\part*{Introduction}
\label{Section1}


The Cosmos ecosystem's independent, sovereign app-chains require liquidity and diversification of volatility to foster investment into new projects and create ecosystem growth. NIBIRU fulfills this gap by giving ecosystem participants access to leverage through the implementation of several compostable standalone protocols:

\subsection*{Perpetual Futures}

\subsection*{Automated Market Maker (AMM)} 
A simple DEX that enables ecosystem participants to bring decentralized collateral into NIBIRU.


\subsection*{Stablecoin} 
NUSD is a fractionally-collateralized algorithmic stablecoin that can support backing by both decentralized collateral as well as algorithmic means (the burning and redeeming of the governance token NIBI).

At genesis, NUSD will be 100\% collateralized with no algorithmic backing. In the future, provided governance deems it appropriate to instantiate fractional collateralization, the protocol will allow minting NUSD via a dynamic ratio of allow-listed collateral and NIBI.

The parameters of these 3 applications are tuned via governance using NIBI, the staking and utility token of Nibiru. NIBI empowers a holder to submit and vote on governance proposals, which dictate changes to the Nibiru protocol. NIBI accrue the value generated by the protocol and helps bolster the price-stability of NUSD. Stakers of NIBI receive staking rewards and can receive higher staking rewards for longer lock-ups. 

\subsection*{Two-token economic model}
\begin{itemize}
\item NIBI is the governance token for NIBIRU used to stabilize NUSD. NIBI is volatile, accrues the value generated by the protocol and partially collateralizes NUSD.
\item NUSD is a price stable token with an elastic supply that adjusts based on market demand. NUSD is partially algorithmic and partially collateralized.
\vfill
  
\end{itemize}

% ------------------------
% Perpetual Futures Exchange 
% ------------------------
\twocolumn
\part{Nibiru Perps}\label{x/perp}

% History on perps
Perpetual futures, also called perpetual swaps or "perps", were first proposed by Robert Shiller in 1992 but only became popular when introduced to cryptocurrency markets in 2016 by BitMEX. Perps are the most popular financial instrument in crypto markets, with the trading volume across major centralized exchanges reaching trillions of notional USD value. Decentralized perps currently represent 2-3\% of derivatives volume in comparison with 10-11\% DEX to CEX ratio in spot markets\footnote{ https://www.theblockcrypto.com/data/decentralized-finance/dex-non-custodial/dex-to-cex-spot-trade-volume}.

% Perp XYK
NIBIRU Perpetual Futures Exchange (Nibiru Perps, for short) uses a constant product formula with no real assets stored inside. The tokens are sent to a clearing house, which stores the collateral in a vault and uses the AMM\footnote{AMM here refers to the $x-y-k$ model pioneered by Uniswap} solely for price discovery. The AMM model uses virtual tokens in the constant product formula to calculate the price of the derivative allowing for the use of leverage and removing the need for liquidity providers. The protocol controls the funding payments in NUSD and actively monitors the liquidation and the management of the Insurance Fund. The protocol’s intention is to solve the following problems prevalent in the decentralized perp marketplace. 

Open problems which NIBIRU Perps Exchange seeks to address:
\begin{itemize}
\item \textbf{Minimize latency during periods of high volatility}: Design a liquidation engine that minimizes latency by liquidating positions based on the index price rather than the mark price during periods of volatility\footnote{During periods of volatility, generally arbitrageurs cannot provide liquidity as well as adjust their margin}. 
\item \textbf{Minimize the imbalance in open interest}: Formulate the correct AMM model; to minimize the imbalance between long and short open interest.
\item \textbf{Increase the number of unique traders on the platform} Enable traders, besides arbitrageurs and bots and minimize the insurance fund from taking the other side of the trades.
\item \textbf{Reduce the bleeding of the ecosystem fund}: The objective of the NIBIRU protocol is make sure it keeps the funding rates of the listed perps at parity to all other futures perpetual exchanges and monitor the opportunity for arbitrageurs. 
\end{itemize}


\subsection*{Structure of Nibiru Perpetual Positions}

\begin{enumerate}
\item \textbf{Settlement in NUSD} - Contracts are denominated and settled in NUSD, providing a versatile settlement currency across the stablecoin margined futures across all ecosystem.
\item \textbf{No Expiration Date} - You can hold positions without an expiry date. 
\item \textbf{Clear Pricing Rules} - Each futures contract specifies the base asset's quantity delivered for a single contract. For instance, OSMO/USDC, UMEE/USDC and ATOM/USDC futures contracts represent only one unit of its respective base asset, similar to spot markets. 
\end{enumerate}

\textbf{Funding payments module}: Funding payments are used to incentivize traders to take long/short positions. A time-weighted average price from the DEX pool is taken to compute the mark price. The index price is derived from an oracle. Funding payments are calculated and exchanged between traders hourly on Nibiru.

\textbf{Perpetual Ecosystem Fund}: To protect the perp protocol against unexpected events, Nibiru Protocol stores half of the revenue generated from trading fees in a fund called the PerpEF. This ecosystem fund steps in to cover the negative value from a bankrupt position that was not liquidated in time and to pay funding payments that could correct the skew between the long/short positions.

% ---------------------  Liquidations 

\section{Liquidations}: When traders use leverage on their positions, they become open to liquidation risks. When the value of the

\item \textbf{Liquidation module}: Using leverage traders use NUSD to borrow money from the exchange to purchase an asset. When the underlying value of a trader's perp falls, the derivative asset begins to approach the value of its margin, putting the exchange at risk. To prevent the traders’ position from being under-collateralized, the exchange will proactively liquidate the traders’ position.

NIBURU use partial liquidations; where if the 
ratio between the asset value and the margin is above 2\%, only a quarter of the position will be liquidated. Once the margin ratio falls to 7\%, the total position is liquidated. 

The liquidations are triggered by \textbf{liquidations bots} that earn a small percentage of the remaining position.

\subsection*{Clearing House}

When opening a position, tokens are deposited and locked as \textbf{margin} for the position. Under the hood, these tokens are stored with the \textbf{clearing house}, which uses the virtual pools for price discovery by converting the deposit into virtual assets. 

These virtual assets change the reserves of their corresponding pool, determining the price of the derivative (position) and enabling the use of leverage. The protocol controls the funding payments in NUSD and actively monitors the liquidation and the management of the Ecosystem Fund. The protocol’s intention is to solve the following problems prevalent in the decentralized perp marketplace.


\subsection*{Perp: NIBI Token}
Holders who stake their NIBI tokens can vote on or propose new ideas to improve the perps protocol.  A small percentage of the protocol's NIBI inflation feeds into the Ecosystem Fund. NIBI stakers vote on exchange improvements, parameter alterations, new feature implementations, chain updates, and alterations to reward mechanisms.

\subsection*{Perp VIP Trading Program}

Holders who stake their NIBI tokens can vote on or propose new ideas to improve the perps protocol. 10\% of staked NIBI feeds into the ecosystem fund. The NIBI token acts as a backstop. If the Perp EF is unable to cover unexpected losses, the protocol will mint new NIBI tokens and immediately sell them for collateral to keep the system solvent.

NIBI holders will have the ability to vote on exchange improvements, parameter alterations, new feature implementations, chain updates, and inflationary reward mechanisms. NIBI stakers enjoy a trading fee discount proportional to the amount staked.


\subsection{What are the risks?}

NIBIRU ecosystem is built to promote robust decentralization, permissionless creation of perps. Community members can start trading without the supervision of a central authority. New market proposals will require governance approval for listing and a listing fee in NIBI tokens. 

The permissionless state of market creation can drive the protocol to in-solvency. NIBIRU doesn’t want risk of one market to spill over to other markets, it requires each new pair to establish an insurance fund before trading is allowed to commence.

% ------------------------
% Automated Market Maker (AMM) 
% ------------------------

\part{Automated Market Maker (AMM)}

NIBIRU AMM will use the generalized constant product function. The AMM is a value function which enforces that the product of the asset balances raised to the asset weight in the pool should always remain constant. In the formula, $t$ represents the number of assets in the pool, $\mathcal{Q}_t$ is the quantity of the asset, and $\mathcal{W}_t$ is the asset weight. 

\begin{equation}
\mathcal{Q}_1^{W_1} \times \mathcal{Q}_2^{W_2} \times \cdots \times \mathcal{Q}_t^{W_t} = k
\label{AMM-Math}
\end{equation}

NIBIRU takes the values in (\ref{AMM-Math}) and computes $k$: The goal is to keep $k$ constant by only changing the asset balances while keeping the asset weights constant. 
Each pair of tokens in the pool has a price that is dependent on the balance $\mathcal{Q}$ and weight $\mathcal{W}$ for that specific pair. 

Formally, the price that the swap executes at is computed as the ratio of the token balances normalized by the token weight:

\begin{equation}
Price = \frac{\mathcal{A}_b/\mathcal{A}_w}{\mathcal{B}_b/\mathcal{B}_w}
\label{Price}
\end{equation}
 
In~(\ref{Price}), $\mathcal{A}$ represents the token being sold (going into the pool) while token $\mathcal{B}$ is the token being bought (going out of the pool). If the pool owner doesn’t modify the asset reserves, it is easy to see that the price changes solely based on trades since the asset weights must always remain constant. Equation~(\ref{Price}) ensures that the price of the asset bought increases while the price of the asset sold decreases. The arbitrage opportunities guarantee that the prices offered by the pools in ~(\ref{AMM-Math}) move in conjunction with the rest of the market.

% ------------------------
% Stablecoin 
% ------------------------
\part{NUSD: The Nibiru Stablecoin}


Nibiru implements a fractional-algorithmic stablecoin NUSD. NIBI serves as a volatility-absorbing asset and gives NUSD greater capacity for scaling by helping the system grow with a smaller requirement for exogenous collateral. For launch, we whitelist USDC as the exogenous and NIBI as endogenous collateral.

NUSD’s creation and annihilation mechanism is dependent on the prices obtained from the chain’s price feed. As a result, arbitrage opportunities arise when the price of NUSD falls off peg because traders can profit from secondary markets.

\subsubsection{Collateral Ratio}

The \textbf{collateral ratio (CR)} is defined as the proportion of NUSD value that is transacted as collateral during mints and burns of NUSD. For example, if the collateral ratio is 70\% and a user wants to mint 100 NUSD,  70 NUSD worth of collateral and 30 NUSD worth of NIBI are required. Similarly, if the user instead wants to burn 100 NUSD, she’d receive an identical proportion of NIBI and collateral.

Nibiru’s collateral ratio changes in response to the price of NUSD on the open market. A decrease in collateral ratio helps support protocol expansion when the system is growing, while an increase in collateral during price downturns helps curate system trust. If the price of NUSD goes too far (pre-determined threshold) below its peg, the collateral ratio is increased. And if NUSD goes far enough above its peg, the collateral ratio is decreased. At genesis, the protocol will start out with a CR of 100\%. As users interact with the protocol, the CR will fluctuate based on the demand for NIBI and NUSD.

\subsubsection{Effect of the Liquidity Ratio on the Collateral Ratio}

Nibiru adjusts the Collateral Ratio based on changes in liquidity, measuring NIBI liquidity against the total supply of NUSD. As the liquidity of NUSD increases, NIBI liquidity must increase proportionally to the supply of NUSD, and more NUSD can be redeemed with less of a percentage impact on NIBI supply. As a result, the system can absorb more NIBI sell pressure from NUSD redemptions being sold without risking negative feedback loops.

Nibiru deploys a stability mechanism whereby changes in NUSD supply and NIBI liquidity adapt to the changes in the market and control the amount of endogenous collateral backing the system. During sustained periods of net liquidity ratio change, the market essentially expresses confidence in the endogenous collateral and signals to the protocol that the CR should be lower to better facilitate scaling.

The total value of assets locked in the Nibiru ecosystem will always be less than or equal to the NUSD market cap, so NUSD can more easily adapt to the pace of increasing stablecoin demand. During sustained periods of net negative liquidity ratio change, the market signals that more exogenous collateral should back the system. The exogenous collateral within the system dampens the reflexive downward spirals that are more likely to occur in systems entirely reliant upon endogenous collateral.

The \textbf{worst-case scenario for NUSD} is if NUSD holders can drain all the collateral from the system through redemptions, leaving the remaining holders with improperly collateralized NUSD. While this scenario is theoretically possible, it’s unlikely. The CR is not designed to rapidly fluctuate so there will not be extended opportunities in which the CR vastly exceeds the actual percentage of collateral in the system.

The stability mechanism for NUSD can become more sophisticated over time, alongside the system’s responsiveness to market information. A CR of 0\% implies the market has complete confidence in the endogenous collateral and no desire for the ability to redeem exogenous collateral, whereas a CR of 100\% implies the market has zero confidence in endogenous collateral, hindering growth. Nibiru’s CR dynamically adjusts with the growth of the protocol and emergent market intelligence.

\part{Appendix}

\section{Automated Treasury Management (ATM)}

NUSD will implement Automated Treasury Management (ATM), to deploy the whitelisted exogenous collateral from NUSD minting into various strategies, while maximizing the core stability mechanism, formalizing the accounting of the balance sheet, defining how much NIBI can be bought and burned with profits above that CR. The ATM framework would enable the deployment of the NUSD into an arbitrarily large set of strategies to generate revenue, upgrade the system, and enable partnerships – all without undermining the price stability of NUSD. 

Active ATM would control the NUSD monetary policy and deploy successful strategies: (\textit{a}) 

\subsection{Investor ATM} 
Investor ATM will deploy NUSD whitelisted collateral to yield aggregators and money markets. The ATM will allocate funds into strategies that have a time delay for withdrawals, so it never lowers the CR and can pull out the collateral if needed for NUSD redemptions

\subsection{AMM ATM} 
AMM ATM will deploy idle and newly minted NUSD into the NIBI-NUSD pool. NUSD will utilize this DEX ATM to earn revenue and increase NUSD liquidity as well as strengthen its peg. NIBURU protocol will create the NIBI-NUSD pool setting the parameters of the pool. The ATM would build incentive structure to maximize the LP rewards

\subsection{Lending ATM} 
Lending ATM would let the borrowing of over-collateralized NUSD into money markets gives the ATM the ability to increase/decrease NUSD interest rates through minting and burning. This ATM would control the NUSD adoption by offering highly competitive borrowing rates

\subsection{Liquidation ATM} 

would let the deployment of NUSD to capture the yield from liquidation from the perp and the lending protocols integrated on NIBURU. 

The modular design of the ATMs based on its current state makes NUSD highly adaptive to the market conditions. NIBURU protocol will invest the excess collateral with ATM strategies to increase stability and incentivize adoption of NUSD. NUSD will start with whitelisting IBC assets like UST. Eventually NUSD will utilize a basket of other stablecoins to maximize the decentralization ratio (extent to how much decentralized stablecoin system is dependent upon centralized components). 

%\begin{thebibliography}{00}
%\bibitem[Josep and Saut (1990)] {JosephSaut1990}{\sc Sunny Agarwal.} Staking module: {\it Cosmos}, {\bf 1}, pp.11--22, {\url{https://docs.cosmos.network/master/modules/gov/}}.
%\bibitem[Josep and Saut (1990)] {JosephSaut1990}{\sc Sunny Agarwal.} List of cosmos ecosystem validators. {\it Cosmos}, {\bf 1}, pp.91--127, {\url{https://cosmos.fish/leaderboard/all}}
%\bibitem[Josep and Saut (1990)] {JosephSaut1990}{\sc Sunny Agarwal.} We need to onboard validators by running the validator onboarding program. {\it Akash}, {\bf 1}, pp.12--32, {\url{https://akash.network/validators/}}
%\end{thebibliography}



\end{document}









